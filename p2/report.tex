\documentclass[11pt]{article}
\parindent=0in
\parskip=8pt
\begin{document}
\title{COMP 512 - Project Report - Part 2}
\author{Luke Emery-Fertitta - Student ID: 260569374 \\ Jonathan Campbell - Student ID: 260481285}
\date{2015 November 11}
\maketitle

\section*{System Architecture}

The lock manager is stored at the middleware server. It receives lock requests from the transaction manager, which specifies the lock type (read or write), the data item to lock, and the ID of the transaction requesting the lock. It uses this centralized model so that the resource managers themselves need know nothing about locking. This approach also simplifies the process of waiting on a lock, since the middleware will know not to accept anymore operations from the same transaction until the lock is received (or a deadlock occurs), instead of having to be told by a resource manager who is waiting that it should wait as well. \par

The lock manager uses the strict 2PL locking scheme. Operations that read a data item will request a shared lock on the item. More than one transaction can receive a shared lock for the same data item. Operations that write a data item will request an exclusive lock on the item, which will only be given if there are no shared locks on that item, except for its own. If the transaction does have a shared lock and requests an exclusive lock, the lock is converted to exclusive if possible. If other transactions already have a shared lock on that item, then the requesting transaction will wait until those are released. Transactions will release all their locks only at commit or abort time. \par

The transaction manager is also contained entirely on the middleware server. Again, this design greatly simplifies the message passing requirements. Because the lock manager is local, the only message passing required after a request is received is the data access, which was already implemented in the first part of the project. Upon server initialization, a transaction manager is created. Then, we simply pass off the \texttt{start}, \texttt{abort}, and \texttt{commit} methods from the API directly to the transaction manager. \par

The \texttt{start} method initializes a transaction and provides the client with a transaction identifier to use with all other operations. This method atomically grabs the next available, unique transaction identifier integer, creates a \texttt{Transaction} object, and begins the TTL timer for the transaction. \par

When an operation is sent from the client to the middleware to be performed, the middleware will ask the transaction manager for the necessary locks and also inform it of the operations to perform in case of an abort (the undo operations). These operations are exactly the reverse of the operation sent by the client, e.g., if the operation was removeFlight, the undo operation will be addFlight, with the necessary parameters. The undo operation is passed as a Runnable using the lambda functionality in Java 8. After informing the TM of these details, the middleware will perform the requested operation, forwarding it to a resource manager if appropriate. When it receives the result of the operation, it will check if the operation succeeded or failed - in the latter case, it will ask the TM to remove the last undo operation, since there is nothing to undo. \par

The \texttt{abort} method is available as an API call. Additionally, aborts will automatically be performed upon deadlocks, as a form of deadlock resolution, and TTL expirations, as a form of client timeout handling. The TTL expiration is checked by a \texttt{ScheduledExecutorService}, which is simply a scheduled, concurrent operation that ensures the transaction has been used within a specified time period. In this case, ``used'' refers to any read or write operation which is called with the transaction identifier.  \par

When a transaction is aborted, the transaction manager will run the undo operations that have built up throughout the transaction's lifespan, ask the lock manager to release any locks held by the transaction, and remove the transaction from its transaction list. \par

When a transaction commits, the TM asks the LM to release the transaction's locks, and removes the transaction from its list. \par

The \texttt{shutdown} method is an API call for a full-system, soft shutdown. If the client requests a shutdown, the transaction manager will deny any future start transaction requests and wait for all running transactions to complete. After this, each of the resource managers is sent a shutdown request and the middleware terminates. The resource managers will exit gracefully upon receiving the shutdown message from the middleware. \par


\section*{Problems encountered}

One problem that was encountered that originated from the centralized model was the management of undo functions. Since the middleware contains the transaction manager, it is also responsible for informing the TM of the undo operations to perform in case of abort, before the actual operation takes place. Therefore, the middleware will need to know the reverse of each operation, even when it is only forwarding the operation to a resource manager. The undo operations are hardcoded in the middleware, so knowledge about each operation must reside at both middleware and RMs, which breaks the pattern of functional isolation. Further, some undo operations are more complex, in cases where an operation (like addFlight) can have different effects based on data state (it can add a completely new flight, or edit details of a current flight), leading to a necessity for state analysis on the middleware and communication with the RM to determine which action will be performed, in order to record the correct undo operation.

Concurrency was a concern due to the multithreaded request handling and asynchronus requirements of the transaction manager. We had to ensure that the transaction identifier was unique regardless of \texttt{start} execution order, and that once a shutdown request has been initiated, no client is able to start a new transaction. The former problem was solved with an \texttt{AtomicInteger} attribute, which allows for an atomic get and increment. The latter problem was solved by synchronizing the shutdown and start methods and adding an \texttt{isShutdown} attribute, such that once the shutdown has begun, threads will block at \texttt{start}, and after it finishes, transactions will fail to start.

\section*{Testing}

-Lock Manager test framework - J

-Transactions tests (to be done) - L


\end{document}
